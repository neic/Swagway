\documentclass[a4paper,oneside,article,danish,table]{memoir}
\checkandfixthelayout
\XeTeXtracingfonts= 1
\usepackage{babel,microtype,verbatim,threeparttable,amsmath,amssymb,unicode-math,hyperref,siunitx,mhchem,tikz,pgfplots,tikz-timing}
\sisetup{per-mode=symbol}
\microtypesetup{final,verbose=silent}
\usetikzlibrary{mindmap,arrows,positioning,shapes}
%\setmainfont[Ligatures={TeX}]{Arno Pro}
\setmainfont{Linux Libertine O}
\setmathfont{[Asana-Math]}
%\hfuzz=1pt
\usepackage[margin,draft]{fixme}
\fxusetheme{color}

\newcommand{\authorvar}{Mathias~Dannesbo}
\newcommand{\pretitlevar}{Programmering C eksamen:}
\newcommand{\titlevar}{Swagway Debugger} 
\newcommand{\subtitlevar}{0} 
\newcommand{\datevar}{\today} 
\newcommand{\subjectvar}{Programmering C}
\newcommand{\classvar}{}
\newcommand{\teachervar}{}

\makepagestyle{articlehead}
\makeevenhead{articlehead}{}{\titlevar}{}
\makeevenfoot{articlehead}{}{\thepage}{}
\makeoddhead{articlehead}{}{\titlevar}{}
\makeoddfoot{articlehead}{}{}{\thepage}
\pagestyle{articlehead}

\usepackage{listings,textcomp}
\lstset{language=[Sharp]C,
%  morekeywords={[2]abs,acos,asin,atan,atan2,ceil,constrain,cos,degrees,exp,floor,log,map,max,min,radians,random,randomSeed,round,sin,sq,sqrt,tan,bitRead,bitWrite,bitSet,bitClear,bit,highByte,lowByte,analogReference,analogRead,analogWrite,attachInterrupt,detachInterrupt,delay,delayMicroseconds,digitalWrite,digitalRead,interrupts,millis,micros,noInterrupts,noTone,pinMode,pulseIn,shiftIn,shiftOut,tone,Serial,Serial1,Serial2,Serial3,begin,end,peek,read,print,println,available,flush,setTimeout,find,findUntil,parseInt,parseFloat,readBytes,readBytesUntil},
  keywordstyle={\bfseries\color[rgb]{0,0,1}},
%  keywordstyle={[2]\bfseries\color[rgb]{0.8,0.33,0}},
  identifierstyle=\ttfamily,
  commentstyle=\color[rgb]{0.133,0.545,0.133},
  stringstyle=\color[rgb]{0.627,0.126,0.941},
  showstringspaces=false,
  basicstyle=\small\ttfamily,
  numberstyle=\footnotesize,
  numbers=left,
  stepnumber=1,
  numbersep=10pt,
  tabsize=2,
  breaklines=true,
  breakatwhitespace=false,
  upquote=true,
  extendedchars=true,
  literate={æ}{{\ae}}1
    {ø}{{\o}}1
    {å}{{\aa}}1
}

\newcommand{\form}[2]{\lstinputlisting[firstnumber=#1,firstline=#1,lastline=#2]{../Software/Form1.cs}}
\newcommand{\boarddate}[1]{\textcolor{blue!80!black}{#1}}
\newcommand{\issue}[1]{\textsuperscript{\textcolor{blue!80!black}{\href{https://github.com/neic/Swagway/issues/#1}{\##1}}}}


\begin{document}
% \includepdf{forside.pdf} \clearpage%------------------------ Forside

\begin{center}
  \if\pretitlevar 0
  \else{\Large\pretitlevar\\} \fi
  \textsc{\HUGE\titlevar\\}
  \if\subtitlevar 0
  \else {\Large\subtitlevar\\} \fi
  %\vspace{1em}
  {\LARGE 
  af\\
   \authorvar}\\
 \datevar\\
\end{center}
\thispagestyle{empty}

\chapter{Indledning}
Formålet med opgaven er at lave et stykke software der kan bruges sammen med forfatterens eksamenprojekt i Teknikfag~A:~El. Formålet med teknikfags projektet er at bygge en Segway klon, en motoriseret selvbalancrende tohjulet trasportenhed. Nawnet på Segway klonen er Swagway. Under udviklingen af Swagway blev der behov for et stykke software til at teste et filter.

\subsubsection{Kort beskrivelse af Swagway funktion}
For at Swagway kan holde sig lodret skal den kende vinklen den er fra lodret. Det udregner den ved at læse data fra to sensorer: et accelerometer og et gyroskob.

Et gyroskob kan måle vinkelhastigheder. Det vil sige ændringer i vinklen. Hvis denne vinkelhastighed integreres over tid finder man vinklen gyroskobet har flyttet sig. Problemet med at integrer gyroskobdata er, at pga. måleusikkerheder vil det målte nulpunkt drive væk fra det fysiske nulpunkt.

Et accelerometer kan måle accelerationer. Man kan med hjælp fra tanges og data fra to akser udregne den vinklen accelerometeret står i forhold til jordens tygdekræft. Problemet med dette er at accelerometer måler alle accelerationer, ikke kun jordens tyngdekræft. Når Swagwayen fx accelerer, bremser eller køre over en sten bliver den udregnet vinkel forkert.

For at komme begge problemer til livs bliver begge sensorer brugt og data fra begge to bliver samlet i et såkaldt Kalman-filter. Kalman-filtret kan tilnærmelses ses som et high-pass og low-pass filter. Det taget de hurtige ændringer (high-pass) fra gyroskob-dataene og bruger accelerometer-dataene til at holde det målte og det fysiske nulpunkt over ens.

\subsubsection{Swagway Debugger}
For at teste om Kalman-filtret er implementeret og tunet korrekt er det nødvendig at sammenligne de tre vinkler fra hhv. gyroskobet, accelerometert og kalmanfiltret. En måde at gøre dette på er ved at sende disse data fra elektronikken på Swagwayen til en PC og vise den grafisk. Det er her Swagway~Debugger-softwaren kommer ind i billedet.

Swagway~Debugger modtager data fra en Arduino mikrocontroller på Swagwayen over en seriel-port og skriver dataene på en graf. Som der står i produktbeskrivelsen:
\begin{quotation}
  Opgaven er at lave software der kan modtage data fra en Arduino over en serielport og visualiserer det. Softwaren er til test af filtereringen af data fra et gyroskop og accelerometer. Softwaren viser bl.a. et rullende plot, samt nogle viserinstrumenter.
\end{quotation}

\chapter{Udførelse}
Swagway~Debugger er skrevet på Windows i C\# som en Windows Form Application. Der er ingen eksterne afhængighedder; det bruger kun Microsoft biblioteker: \fxwarning{Er dette biblioteker?}
\form{9}{12}
Der er fem globale variabler:
\form{18}{23}
De tre første er buffere der bliver brugt til midlertidig at gemme modtaget data i, de to sidste er tællere over antallet af pakker.

\section{Indstillinger}
\fxwarning{Indsæt billed af "settings"-tab}
Ved opstart af applikationen indlæses alle tilgængelige serial-porte og der sættes en standardinstilling for valget af seriel hastighed:
\form{34}{39}
Ved afslutning af applikationen lukkes en eventuel åben seriel-port:
\form{41}{48}

\texttt{LoadSerialPorts()} kaldes ved opstart og af knappet \texttt{btReload}. Den føjer alle systemets serial-porte til listen, sætter den første seriel-port som standardindstilling og advare i statuslinjen hvis der ikke er fundet seriel-porte:
\form{50}{68}

% btReload_Click undladt
Eventhandleren for \texttt{btConnect} både forbinder og afbryder til en seriel-port. Hvis der er afbrud finder den hvilken serielport og hastighed der er valgt, forbinder og opdaterer statuslinjen og dens egen tekst. Hvis der er oprettet forbindelse afbryder den og ligeledes opdaterer statuslinjen og dens egen tekst:
\form{76}{104}
Up-Down tælleren%fix
\texttt{udPackages} bestemmer hvor meget data der skal vises på grafen. Når værdien i tælleren ændres ændre den dynamisk skala. Endeligt kalder opdaterer den grafens X-akses maksimal værdi:
\form{106}{125}

\section{Seriel modtagelse og monitor}
\fxwarning{Vis monitor}
Når der, efter der er forbundet til en seriel-port, ankommer data lægges det i bufferen \texttt{readFromUART} og \texttt{ReadToMonitor} kaldes:
\form{131}{136}
\texttt{ReadToMonitor} Skriver den modtaget data, \texttt{readFromUART}, direkte til monitoren uden at vidrebehandle den. Den undesøger om der er meget data i monitoren og sletter en smugle af det ældste hvis det er tilfældet. Til slut kalder den \texttt{CleanData()}.
\form{138}{149}
Datapakkerne fra Swagwayen er af formen $<\langle \text{gyro} \rangle,\langle \text{acc} \rangle,\langle \text{kalm} \rangle,\langle \text{tid} \rangle>$
\begin{tabbing}
H\=vor\\
\> $\langle \text{gyro} \rangle$ er vinklen målt med gyroskobet med to decimaler og eventuelt negativt fortegn,\\
\> $\langle \text{acc} \rangle$ er vinklen målt med accelerometret med to decimaler og eventuelt negativt fortegn,\\
\> $\langle \text{gyro} \rangle$ er vinklen udregnet af Kalman-filtret  med to decimaler og eventuelt negativt fortegn og\\
\> $\langle \text{tid} \rangle$ er tiden i µS siden sidste pakke blev sendt.
\end{tabbing}
\texttt{CleanData()} tager den rå modtaget data, \texttt{readFromUART}, og tilføjer det til en ny buffer, \texttt{rxStringBuffer}, som allerede indeholder en rest data fra sidste gang. Denne buffer bliver splittet og hver pakke bliver indsat i listen \texttt{rxListBuffer}. Den sidste, ikke fuldstændige, pakke bliver lagt tilbage i \texttt{rxStringBuffer} og er klar til at bliver parret sammen ved næste kald af \texttt{CleanData()}:
\form{151}{157}

%\chapter{\texttt{Debug\_test.ino}}
\lstset{language=[Visual]C++,
  morekeywords={[2]abs,acos,asin,atan,atan2,ceil,constrain,cos,degrees,exp,floor,log,map,max,min,radians,random,randomSeed,round,sin,sq,sqrt,tan,bitRead,bitWrite,bitSet,bitClear,bit,highByte,lowByte,analogReference,analogRead,analogWrite,attachInterrupt,detachInterrupt,delay,delayMicroseconds,digitalWrite,digitalRead,interrupts,millis,micros,noInterrupts,noTone,pinMode,pulseIn,shiftIn,shiftOut,tone,Serial,Serial1,Serial2,Serial3,begin,end,peek,read,print,println,available,flush,setTimeout,find,findUntil,parseInt,parseFloat,readBytes,readBytesUntil},
    keywordstyle={[2]\bfseries\color[rgb]{0.8,0.33,0}}}
%\lstinputlisting{../Software/Debug_test/Debug_test.ino}

\end{document} 