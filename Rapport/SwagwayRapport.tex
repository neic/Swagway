\documentclass[a4paper,oneside,article,danish,table,draft]{memoir}
\XeTeXtracingfonts= 1
\usepackage{babel,microtype,verbatim,amsmath,amssymb,unicode-math,ulem,url,siunitx,mhchem,tikz,pgfplots}
\sisetup{per-mode=symbol}
\microtypesetup{final,verbose=silent}
\usetikzlibrary{mindmap,arrows,positioning,shapes}
\setmainfont[Ligatures={TeX}]{Arno Pro}
\setmathfont{Arno Pro}
\setmathfont{[Asana-Math]}
%\hfuzz=1pt
\usepackage[margin,draft]{fixme}
\fxusetheme{color}

\newcommand{\authorvar}{Carl~Emil~Grøn~Christensen and Mathias~Dannesbo}
\newcommand{\pretitlevar}{0}
\newcommand{\titlevar}{Swagway} 
\newcommand{\subtitlevar}{0} 
\newcommand{\datevar}{\today} 
\newcommand{\subjectvar}{Teknikfag~A:~El}
\newcommand{\classvar}{}  
\newcommand{\teachervar}{}

\makepagestyle{articlehead}
\makeevenhead{articlehead}{}{\titlevar}{}
\makeevenfoot{articlehead}{}{}{\thepage}
\makeoddhead{articlehead}{}{\titlevar}{}
\makeoddfoot{articlehead}{}{}{\thepage}


\begin{document}
% \includepdf{forside.pdf} \clearpage%------------------------ Forside

\begin{center}
  \if\pretitlevar 0
  \else{\Large\pretitlevar\\} \fi
  \textsc{\HUGE\titlevar\\}
  \if\subtitlevar 0
  \else {\Large\subtitlevar\\} \fi
  %\vspace{1em}
  {\LARGE 
  af\\
   \authorvar}\\
 \datevar\\
\end{center}

\vfill
\begin{abstract} %------------------------------ Abstract
\end{abstract}\vfill
\noindent
\begin{tabular*}{\textwidth}{@{\extracolsep{\fill}} ll}

\end{tabular*}

\thispagestyle{empty}
\clearpage
\setcounter{tocdepth}{2} \tableofcontents \clearpage


\chapter{Indledning}\label{chap:ind}
\section{Problemformulering}

\section{Indput}

\subsection{Sensor}
I2C, Pull-up, Bus capasistance, level shifter,

\section{Control}

\subsection{Sensor læsning}

\subsection{Filter}
\subsubsection{Komplimentær filter}
\subsubsection{Kalman filter}
\subsubsection{Modificeret kalman filter}

\section{Output}

\subsection{Motorstyring}
\begin{table}[htbp]
  % \caption[]{}
  \centering
  \begin{tabular}{ccc|cccc|ccccl}
      \toprule
     P2&P3&P5 &Q1&Q2&Q3&Q4 &Q1&Q2&Q3&Q4\\
     P4&P6&P9 &Q5&Q6&Q7&Q8 &Q5&Q6&Q7&Q8\\
      \midrule
     0&0&0 &0&0&0&0 &1&0&1&0 & Brake\\
     1&0&0 &1&1&0&0 &0&1&1&0 & $\circlearrowleft$\\
     0&1&0 &0&0&0&1 &1&0&1&1 & Short\\
     1&1&0 &1&1&0&1 &0&1&1&1 & Short\\
     0&0&1 &0&0&1&0 &1&0&0&0 & Off ($\circlearrowright$)\\
     1&0&1 &1&1&1&0 &1&1&0&0 & Off ($\circlearrowleft$)\\
     0&1&1 &0&0&1&0 &1&0&0&1 & $\circlearrowright$\\
     1&1&1 &1&1&1&1 &0&1&0&1 & Brake\\
    \end{tabular}
    % \label{tab:}
  \end{table}

H-bro, PWM, PWM-kondensator, beskyttelses dioder, 4000 serie, optocopler



\chapter{Konklusion} \label{chap:kon}
\clearpage
\listoftables
\listoffigures
% \nocite{*}
\bibliographystyle{dk-apali} \bibliography{bib}
\clearpage \appendix

\chapter{Status log}

\section{13. marts}
Mainbord er fungerende. v2.0 af motorboardet er næsten færdig.

Kredsløbet uden om printne er næsten færdig.

Vi kan læse data fra IMUen og vi har et halvt implementert kalman-filter.

Efter kalmanfilteret fungere skal der implementeres PID med wrapper kode.

\end{document}